\documentclass[a4paper,12pt]{article}
\usepackage{amsmath, bm}
\usepackage[margin=1in]{geometry}
\setlength\parindent{0pt}
\usepackage{hyperref}
\usepackage{xcolor}
\hypersetup{
    colorlinks,
    linkcolor={blue},
    citecolor={blue}
}
\usepackage{appendix}

\usepackage{listings}
\usepackage{color}
 
\definecolor{codegreen}{rgb}{0,0.6,0}
\definecolor{codegray}{rgb}{0.5,0.5,0.5}
\definecolor{codepurple}{rgb}{0.58,0,0.82}
\definecolor{backcolour}{rgb}{0.95,0.95,0.92}

\usepackage{courier}

\lstdefinestyle{mystyle}{
    backgroundcolor=\color{backcolour},   
    commentstyle=\color{codegreen},
    keywordstyle=\color{magenta},
    numberstyle=\tiny\color{codegray},
    stringstyle=\color{codepurple},
    basicstyle=\footnotesize\ttfamily,
    breakatwhitespace=false,         
    breaklines=true,                 
    captionpos=b,                    
    keepspaces=true,                 
    numbers=left,                    
    numbersep=5pt,                  
    showspaces=false,                
    showstringspaces=false,
    showtabs=false,                  
    tabsize=2
}
 
\lstset{style=mystyle}


%Gummi|065|=)
\title{\textbf{Assignment 3}}
\author{Kalpesh Krishna (140070017) \\ Kumar Ayush (140260016)}
\date{}
\begin{document}

\maketitle


\section{Question 1}
\subsection{Part a) and b)}
We obtain the following output on running \texttt{memory.c},
\begin{lstlisting}
kalpesh@kalpesh-Inspiron-3542:assign3$ ./a.out 
bss 1 - 0x601054
bss 2 - 0x601058
data segment 1 - 0x601048
data segment 2 - 0x60104c
stack 1 - 0x7ffd3dd72458
stack 2 - 0x7ffd3dd7245c
heap 1 - 0xf57010
heap 2 - 0xf57030
heap 1 (pointer) - 0x7ffd3dd72460
heap 2 (pointer) - 0x7ffd3dd72468
\end{lstlisting}
This is consistent with the output of \texttt{objdump -h a.out},
\begin{lstlisting}
 12 .text         00000262  0000000000400490  0000000000400490  00000490  2**4
                  CONTENTS, ALLOC, LOAD, READONLY, CODE
...
 23 .data         00000018  0000000000601038  0000000000601038  00001038  2**3
                  CONTENTS, ALLOC, LOAD, DATA
 24 .bss          00000010  0000000000601050  0000000000601050  00001050  2**2
                  ALLOC
\end{lstlisting}
\subsection{Part c)}
\begin{itemize}
\item \textbf{Stack} - As expected, the variables have been allocated on contiguous memory locations on the stack (they are of type \texttt{int} and each have size 4 bytes)
\item \textbf{Heap} - The two integer variables are not on contiguous memory locations since we used two separate \texttt{malloc()} calls. Notice that the addresses are much smaller when compared to the stack addresses, indicating that the heap generally occupies the lower address spaces.
\item \textbf{bss} - The addresses we obtain are contiguous and consistent within the \texttt{bss} segment range from 0x601050 to 0x601060 (exclusive).
\item \textbf{data segment} - The addresses we obtain are contiguous and consistent within the \texttt{data} segment range from 0x601038 to 0x601050 (exclusive).
\end{itemize}
\subsection{Part d)}
In this part we allocate exactly 32 bytes (eight integers) in each of the segments. The first \texttt{size} call is before the allocation and the second \texttt{size} call is after allocation.
\begin{lstlisting}
kalpesh@kalpesh-Inspiron-3542:assign3$ gcc memory2.c 
kalpesh@kalpesh-Inspiron-3542:assign3$ size ./a.out 
   text	   data	    bss	    dec	    hex	filename
   1115	    552	      8	   1675	    68b	./a.out
kalpesh@kalpesh-Inspiron-3542:assign3$ gcc memory2.c 
kalpesh@kalpesh-Inspiron-3542:assign3$ size ./a.out 
   text	   data	    bss	    dec	    hex	filename
   1115	    584	     40	   1739	    6cb	./a.out
kalpesh@kalpesh-Inspiron-3542:assign3$ 
\end{lstlisting}
Notice both the \texttt{data} and \texttt{bss} segments have increased by 32 bytes. Uninitialized global or static variables are allocated in the \texttt{bss} segment whereas initialized global or static variables are allocated in the \texttt{data} segment.
\subsection{Part e)}
We obtain the virtual address section of the text segment using the \texttt{objdump -h} command.
\begin{lstlisting}
kalpesh@kalpesh-Inspiron-3542:assign3$ objdump -h memory | grep 'text'
 12 .text         00000272  0000000000400490  0000000000400490  00000490  2**4
                  CONTENTS, ALLOC, LOAD, READONLY, CODE
 13 .fini         00000009  0000000000400704  0000000000400704  00000704  2**2
                  CONTENTS, ALLOC, LOAD, READONLY, CODE
                  
kalpesh@kalpesh-Inspiron-3542:assign3$ objdump -h memory2 | grep 'text'
 12 .text         00000172  0000000000400400  0000000000400400  00000400  2**4
                  CONTENTS, ALLOC, LOAD, READONLY, CODE
 13 .fini         00000009  0000000000400574  0000000000400574  00000574  2**2
                  CONTENTS, ALLOC, LOAD, READONLY, CODE

kalpesh@kalpesh-Inspiron-3542:assign3$
\end{lstlisting}
For the first program, the size is 0x272 bytes, or 626 bytes. The text segment begins at 0x400490 and ends at 0x400704 (exclusive). 2 extra bytes are needed due to alignment (\href{https://stackoverflow.com/questions/47186199/empty-space-between-text-and-fini-data-segments}{reference}).\\
Similarly, for the second program, the size is 0x172, or 370 bytes. The text segment begins at 0x400400 and ends at 0x400574 (exclusive). 2 extra bytes are needed due to alignment.
\pagebreak
\section{Question 2}
The \texttt{/proc} filesystem contains a map between virtual memory pages and physical memory pages in the \texttt{/proc/\$(PID)/pagemap} and \texttt{/proc/\$(PID)/maps} file. This map can be used to obtain a mapping from virtual addresses to physical addresses.
\end{document}